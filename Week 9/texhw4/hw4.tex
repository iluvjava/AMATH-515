\documentclass[]{article}
\usepackage{amsmath}\usepackage{amsfonts}
\usepackage[margin=1in,footskip=0.25in]{geometry}
\usepackage{mathtools}
\usepackage{hyperref}
\hypersetup{
    colorlinks=true,
    linkcolor=blue,
    filecolor=magenta,
    urlcolor=cyan,
}
\usepackage[final]{graphicx}

% \usepackage{wrapfig}
\graphicspath{{.}}

\begin{document}
\hspace{-1.8em}
Name: Hongda Li
\\
Class: AMATH 515

\section*{Problem (1)}
    \textbf{Objective:} Prove the identity for $\alpha \in \mathbb{R}$:
    $$
        \left\Vert
            \alpha x + (1 - \alpha)y
        \right\Vert^2 + \alpha(1 - \alpha)\left\Vert
            x - y
        \right\Vert^2 = \alpha\left\Vert
            x
        \right\Vert^2 +(1 - \alpha)\left\Vert
            y
        \right\Vert^2
    $$
    Compute it directly from left to right: 
    \begin{align*}\tag{1}\label{eqn:1}
    &\left\Vert
        \alpha x + (1 - \alpha)y
    \right\Vert^2
    + 
    \alpha(1- \alpha) \left\Vert
        x - y
    \right\Vert
    \\
    & \alpha^2 \left\Vert
        x
    \right\Vert^2 
    + 
    (1 - \alpha)^2 \left\Vert
        y
    \right\Vert^2
    +
    2\alpha(1 - \alpha)x^Ty + \alpha(1 - \alpha)(\left\Vert
        x
    \right\Vert^2 + \left\Vert
            y
    \right\Vert^2 - 2x^Ty)
    \\
    &= 
    \Vert x\Vert^2((\alpha^2 + \alpha(1 - \alpha)) 
    + 
    \Vert y\Vert^2 ((1 - \alpha)^2 + \alpha(1 - \alpha))
    + 
    x^Ty(2\alpha(1 - \alpha) - 2\alpha(1 - \alpha))
    \\
    &= 
    \Vert x\Vert^2(\alpha^2 + \alpha - \alpha^2) 
    + 
    \Vert y\Vert^2 ((1 - \alpha^2)+ \alpha - \alpha^2)
    + 
    x^Ty(0)
    \\
    &=\alpha \Vert x\Vert^2 + (1 - \alpha)\Vert y\Vert^2
    \end{align*}
\section*{Problem (2)}
    \subsection*{(a)}
        \textbf{Objective: }Show that $T_\lambda$ and $T$ has the same fixed points. $x$ is a fixed point of the non-expansive operator $T$ if $x \in Tx$, then we want to show that picking any $x\in Tx\implies x\in T_\lambda x$
        \\
        $T_\lambda$ is defined to be: $T_\lambda:= (1 - \lambda)I - \lambda T$. 
        \\
        Consider: 
        \begin{align*}\tag{2a1}\label{eqn:2a1}
        T_\lambda(Tx) 
        &= (1 - \lambda)Tx + \lambda Tx
        \\
        &= Tx - \lambda Tx + \lambda Tx
        \\
        &=Tx
        \end{align*}
        If $x$ is a fixed for for $T$, then $x \in Tx$ which means $T_\lambda(Tx) \implies T_\lambda(x) = Tx $ for any $x$ so $x = T_\lambda(x)$.
    \subsection*{(b)}
        \textbf{Objective:} With that assumption that $\bar{z}$ is the fixed point of the operator $T$, show that: 
        $$
            \Vert T_\lambda z - \bar{z}\Vert^2 
            \le
            \Vert z - \bar{z}\Vert^2 
            -
            \lambda(1 - \lambda) \Vert z - Tz\Vert^2
        $$
        Because of fixed points we know that $\Vert T_\lambda z - \bar{z}\Vert = \Vert T_\lambda(z - \bar{z})\Vert$ then by definition of the $T_\lambda$: 
        \begin{align*}\tag{1b1}\label{eqn:1b1}
            \Vert T_\lambda(z - \bar{z}) \Vert^2
            &= 
            \Vert (1 - \lambda)
                \underbrace{(z - \bar{z})}_{y} + \lambda 
                \underbrace{T(z - \bar{z})}_{x}
            \Vert^2
            \\
            &= \Vert (1 - \lambda)y + \lambda x \Vert^2
            \\
            \underset{\text{Using Problem 1}}{\implies}
            &= - \lambda(1 - \lambda) \Vert x - y\Vert^2 + \lambda \Vert x\Vert^2 + (1 - \lambda)\Vert y\Vert^2
        \end{align*}
        Let's pause for a moment and consider: 
        \begin{align*}\tag{2b2}\label{eqn:2b2}
            y - x &= z - \bar{z} - T(z - \bar{z})
            \\ 
            &=z - \bar{z} - Tz - T\bar{z} 
            \\
            &= z - \bar{z} - Tz - \bar{z} 
            \\
            &= z - T(z)
        \end{align*}
        Let's continue on latest step from \hyperref[eqn:1b1]{1b1}: 
        \begin{align*}\tag{1b3}\label{eqn:1b3}
            \Vert T_\lambda z - \bar{z}\Vert^2
            &=-\lambda(1 - \lambda) \Vert z - Tz\Vert^2 
            + 
            \lambda \Vert x\Vert^2 + (1 - \lambda)\Vert y\Vert^2
            \\
            &= 
            -\lambda(1 - \lambda) \Vert z - Tz\Vert^2 
            + 
            \underbrace{\lambda \Vert T(z - \bar{z})\Vert^2}_{\le\lambda \Vert z - \bar{z}\Vert}
            + (1 - \lambda)
            \Vert z - \bar{z}\Vert^2
            \\
            &\le 
            -\lambda(1 - \lambda) \Vert z - Tz\Vert^2 
            + \lambda \Vert z - \bar{z}\Vert^2 + (1 - \lambda)\Vert z - \bar{z}\Vert^2
            \\
            &= 
            \Vert z - \bar{z}\Vert^2 -\lambda(1 - \lambda) \Vert z - Tz\Vert^2 
        \end{align*}
        It has been shown. 
\section*{Problem (3)}
    \subsection*{(a)}
        \textbf{Objective: } If $T$ is a firmly non-expansive operator, then we want to show that: 
        $$
            \Vert Tx - Ty\Vert^2
            + \Vert (I - T)x - (I - T)y\Vert^2
            \le \Vert x - y\Vert^2
            \iff
            \langle x- y, Tx - Ty\rangle \ge 
            \Vert Tx - Ty\Vert^2
        $$
        Let's focus on one of the terms that is one the LHS: 
        \begin{align*}\tag{3a1}\label{eqn:3a1}
            &=\Vert (I - T)x - (I - T)y\Vert^2 
            \\
            &=\Vert x - Tx - y + Ty\Vert^2
            \\
            &= \Vert x - Tx - y + Ty\Vert^2
            \\
            &=
            \Vert x - y + Ty - Tx\Vert^2
        \end{align*}
        Now, notice that I can move this term to the RHS to the definition of the non-expansive operator, and then we will have: 
        \begin{align*}\tag{3a2}\label{eqn:3a2}
            &\Vert x - y\Vert^2 - \Vert (x - y) - (Ty - Tx)\Vert^2
            \\
            &= 
            \Vert x - y\Vert^2 - \left\lbrack
                \Vert x - y\Vert^2
                + 
                \Vert Ty - Tx\Vert^2
                + 
                2(x - y)^T(Ty - Tx)
            \right\rbrack
            \\
            &= 
            -\Vert Ty - Tx\Vert^2 - 2(x - y)^T(Ty - Tx)
        \end{align*}
        So then, we have the expression: 
        \begin{align*}\tag{3a3}\label{eqn:3a3}
            & -\Vert Ty - Tx\Vert^2 - 2(x - y)^T(Ty - Tx) \ge 
            \Vert Tx - Ty\Vert^2
            \\
            &\iff -2(x - y)^T(Ty - Tx) \ge 2 \Vert Tx - Ty\Vert^2
            \\
            &\iff 
            (x - y)^T(Tx - Ty) \ge \Vert Tx - Ty\Vert^2
            \\
            &
            \iff 
            \langle x - y, Tx - Ty\rangle 
            \ge 
            \Vert Tx - Ty\Vert^2
        \end{align*}
    \subsection*{(b)}
        \textbf{Objective: }Show that: 
        $$
            \langle Tx - Ty, (I - T)x - (I - T)y\rangle \ge 0
        $$
        For this part we can choose to continue what we derived before and them show the equivalency of this statement and the previous statement. 
        \begin{align*}\tag{3b1}\label{eqn:3b1}
            \langle x - y, Tx - Ty\rangle &\ge \Vert Tx - Ty\Vert^2
            \\
            \langle x - y, Tx - Ty\rangle &\ge \langle Tx - Ty, Tx - Ty\rangle
            \\
            \langle x - y, Tx - Ty\rangle 
            -
            langle Tx - Ty, Tx - Ty\rangle &\ge 0
            \\
            \langle Tx - Ty, x -y - Tx + Ty\rangle \ge 0
            \\
            \langle Tx - Ty, (I - T)x + (I - T)y\rangle &\ge 0
        \end{align*}
    \subsection*{(c)}
        \textbf{Objective: } Suppose that $S = 2T - I$, then we let: 
        $$
        \mu = \Vert Tx  - Ty\Vert^2 +
        \underbrace{\Vert (I - T)x - (I - T)y\Vert^2 }_{(1)}
        - \Vert x - y\Vert^2
        $$
        $$
        v = \Vert Sx - Sy\Vert^2 - \Vert x - y\Vert^2
        $$
        Then sow that $2\mu = v$
        \\
        Consider the expression (1), we should have: 
        \begin{align*}\tag{3c1}\label{eqn:3c1}
            & \Vert (I - T)x - (I - T)y\Vert^2
            \\
            &= 
            \Vert x - Tx - y + Ty\Vert
            \\
            &= 
            \Vert (Ty - Tx) + (x - y)\Vert^2
            \\
            &=
            \Vert Ty - Tx\Vert^2 + \Vert x - y\Vert^2
            + 2(x - y)^T(Ty - Tx)
        \end{align*}
        And notice that substituting in we have: 

        \begin{align*}\tag{3c2}\label{eqn:3c2}
            \mu &= 2 \Vert Ty - Tx\Vert^2
            + 2(x - y)^T(Ty - T x)
            \\
            2\mu &= 4 \Vert Ty - Tx\Vert^2 
            + 4(x - y)^T(Ty - T x)
            \\
            &= \Vert 2Ty - 2Tx\Vert^2 + 2(x - y)^T(2Ty - 2Tx) + \Vert x - y^2 \Vert - \Vert x - y\Vert^2
            \\
            &= \Vert 2Ty - 2Tx + x - y\Vert^2 - \Vert x - y\Vert^2
            \\
            & =\Vert 2Tx - 2Ty + y - x\Vert^2 - \Vert x - y\Vert^2
            \\
            &= \Vert Sx - Sy\Vert^2 - \Vert x - y\Vert^2
            \\
            &= v
        \end{align*}
        By the non-expansive property for the operator $S$ we have: 
        \begin{align*}\tag{3c3}\label{eqn:3c3}
            v &= \Vert Sx - Sy \Vert^2 - \Vert x - y \Vert^2 \le 0
            \\
            2\mu &\le 0
            \\
            \mu &\le 0
            \\
            \Vert Tx - Ty\Vert^2 + \Vert (I - T)x - (I - T)y\Vert^2 &\le  \Vert x - y\Vert^2
        \end{align*}
        Therefore, we can conclude that, if the operator $S$ is non expansive, meaning that $2T -I$ is non-expansive, the operator $T$ is firmly non-expansive. 

        
\end{document}