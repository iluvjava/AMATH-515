\documentclass[]{article}
\usepackage{amsmath}\usepackage{amsfonts}
\usepackage[margin=1in,footskip=0.25in]{geometry}
\usepackage{mathtools}
\usepackage{hyperref}
\hypersetup{
    colorlinks=true,
    linkcolor=blue,
    filecolor=magenta,
    urlcolor=cyan,
}
\usepackage[final]{graphicx}

% \usepackage{wrapfig}
\graphicspath{{.}}

\begin{document}
\section*{Problem (1)}
    \subsection*{(a)}
        \textbf{Objective: }Find the conjugate of $\delta_{\mathbb{B}_\infty}(x)$. The function has definition: 
        $$
        \delta_{\mathbb{B}_\infty} = 
        \begin{cases}
            0 & \Vert x\Vert_\infty \le 1
            \\
            \infty & \text{else}
        \end{cases}
        $$
        The definition of conjugate is: 
        \begin{align*}\tag{1a1}\label{eqn:1a1}
            \delta_{\mathbb{B}_\infty}^*(z) 
            &= 
            \sup_x \left\lbrace
                z^Tx - \delta_{\mathbb{B}_\infty}(x)
            \right\rbrace
            \\
            &= \sup_{\Vert x\Vert_\infty} \left\lbrace
            z^Tx
            \right\rbrace
            \\
            &= \max_{x_i}\left\lbrace
                \sum_{i = 1}^n z_i x_i: |x_i|\le 1
            \right\rbrace
            \\
            &= \sum_{i = 1}^n |z_i|
            \\
            &= \Vert z\Vert_1
        \end{align*}

    \subsection*{(b)}
        \textbf{Objective: }Find conjugate for indication function of the 2 norm. 
        From the definition of the function we have: 
        $$
        \delta_{\mathbb{B}_2}(x) = 
        \begin{cases}
            0 & \Vert x\Vert_2 \le 1 
            \\
            \infty & \text{else}
        \end{cases}
        $$
        Then by definition of the conjugate we have: 
        \begin{align*}\tag{1b1}\label{eqn:1b1}
            \delta_{\mathbb{B}_2}^*(z) &= 
            \sup_x \left\lbrace
                z^Tx - \delta_{\mathbb{B}_2}(x)
            \right\rbrace
            \\
            &= 
            \sup_{\Vert x\Vert_2 \le 1} \left\lbrace 
            z^Tx
            \right\rbrace
            \\
            &= \frac{z^Tz}{\Vert z\Vert_2}
            \\
            &= \Vert z\Vert_2
        \end{align*}

    \subsection*{(c)}
        \textbf{Objective: }Find conjugate of $\exp(x)$, assuming vectorized exponential. Notice that the function is smooth and concave and it's univariate. 
        \begin{align*}\tag{1c1}\label{eqn:1c1}
            \exp^*(z) &= \sup_{x} \left\lbrace
            zx - \exp(x)
            \right\rbrace
        \end{align*}
        Assume $z > 0$ consider: 
        \begin{align*}\tag{1c2}\label{eqn:1c2}
            \partial_x[zx - \exp(x)] &= 0
            \\
            [z - \exp(x)] &= 0
            \\
            x &= \log(z)
            \\
            \implies  \exp^*(z) &= \sup_{x} \{zx - \exp(x)\} = z\ln(z) - z
        \end{align*}
        Assume $z = 0$, we have: 
        \begin{equation*}\tag{1c3}\label{eqn:1c3}
            \exp^*(0) = \sup_x \{- \exp(x)\} = 0
        \end{equation*}
        Assume $z < 0$, we have: 
        \begin{equation*}\tag{1c4}\label{eqn:1c4}
            \exp^*(z) = \sup_x \{- \exp(x)\} = \infty
        \end{equation*}
        This is because $\lim_{x\rightarrow -\infty} (zx - \exp(x))$ gives $\infty$ if $z < 0$
        Therefore:
        $$
            \exp^*(z) = \begin{cases}
                z\ln(z) - z & z > 0
                \\
                0 & z = 0 
                \\
                \infty & z < 0
            \end{cases}
        $$

    \subsection*{(d)}
        \textbf{Objective:} Find the conjugate of the function $f(x)=\ln(1 + \exp(x))$. Notice that the function is convex and smooth, therefore, the inverse of it will be concave, hence setting the derivative to zero will provide us with the optimal. 
        \\
        By definition of conjugacy we have: 
        $$
            f^*(z) = \sup_x \left\lbrace
            zx - \ln(1 + \exp(x))
            \right\rbrace
        $$
        consider the partial derivative wrt to x: 
        \begin{align*}\tag{1d1}\label{eqn:1d1}
           \partial_x [zx - \ln(1 + \exp(x))] &= 0
           \\
           z &= \frac{\exp(x)}{1 + \exp(x)}
           \\
           z(1 + \exp(x)) &= \exp(x)
           \\
           z + z\exp(x) &= \exp(x)
           \\
           z &= (1 - z)\exp(x)
           \\
           \frac{z}{1 - z} &= \exp(x)
           \\
           x &= \ln
           \left(
               \frac{z}{1 - z}
           \right)  \quad \text{Substitute this back}
           \\
           \implies
           f^*(z) &= 
           z\ln \left(
               \frac{z}{1 - z}
           \right)
           - 
           \ln \left(
               1 + \frac{z}{1 - z}
           \right)
        \end{align*}
        It's not all sunshine and rainbow, we have some conditions for $z$: 
        \begin{equation*}\tag{1d2}\label{eqn:1d2}
            \text{sign}(z) = \text{sign}(1 - z) \implies x \in (0, 1)
        \end{equation*}
        We can try to cover 2 of the edge cases of when $z = 0$ and $z = 1$ first, notice that this means we have: 
        \begin{align*}\tag{1d3}\label{eqn:1d3}
            \sup_x\{x - \ln(1 + \exp(x))\} &= 0
            \\
            \text{Notice: } \ln(1 + \exp(x)) &> x \quad \forall x
            \\
            \lim_{x\rightarrow\infty}(x - \ln(1 + \exp(x))) & = \lim_{x\rightarrow\infty}(\ln(\exp(x)) - \ln(1 + \exp(x)))
            \\
            &= \lim_{x\rightarrow \infty} \left(
                \ln 
                    \left(
                        \frac{\exp(x)}{1 + \exp(x)}
                    \right)
            \right) = 0
        \end{align*}
        It's bounded above by zero. 
        Similarly, when $z = 0$, we have: 
        \begin{align*}\tag{1d4}\label{eqn:1d4}
            \sup_x \{-\ln(1 + \exp(x))\} & = 0
        \end{align*}
        This is bounded above by zero, and $x\rightarrow-\infty$, this approaches zero. 
        \begin{enumerate}
        \item[1.] Consider $z < 0$ then $\sup_x\{zx - \ln(1 + \exp(x))\} = \infty$ because as $x\rightarrow -\infty$, the expression is approaching infinity. 
        \item[2.] Consider $z > 1$ then $\sup_x\{zx - \ln(1 + \exp(x))\} = \infty$ because: 
        \begin{equation*}\tag{1d5}\label{eqn:1d5}
            \lim_{x\rightarrow\infty}
            \left(
                zx - \ln(1 + \exp(x))
            \right)
            =
            \lim_{x\rightarrow\infty}
            \left(
                \ln(\exp(zx)) - \ln(1 + \exp(x))
            \right)
            =
            \lim_{x\rightarrow\infty}
            \left(
                \ln
                \left(
                    \frac{\exp(zx)}{1 + \exp(x)}
                \right)
            \right) = \infty
        \end{equation*} 
        \end{enumerate}
        Summing all the cases up we have the conjugate function to be: 
        \begin{equation*}\tag{1d6}\label{eqn:1d6}
            f^*(x) = 
            \begin{cases}
                z \ln \left(
                    \frac{z}{1 - z}
                \right)
                -
                \ln \left(
                    \frac{1}{1 - z} 
                \right) & z \in (0, 1)
                \\
                0 & z = 1 \vee z = 0
                \\
                \infty & \text{else}
            \end{cases}
        \end{equation*}
        
    \subsection*{(e)}
        \textbf{Objective: }Look for the conjugate of $f(x):= x\ln(x)$. Notice that the function is convex and smooth. The conjugate is defined to be: 
        $$
        f^*(x) = \sup_x \left\lbrace
        z^Tx - x\ln(x)
        \right\rbrace
        $$
        And noticing that we can take thd derivative, setting it to zero and then solve for $x$, so we can get the value coming out of sup, we treat this function as a univariate function. 
        \begin{align*}\tag{1e1}\label{eqn:1e1}
            \partial_x[zx - x\ln(x)] &= 0
            \\
            z - (\ln(x) + 1) &= 0
            \\
            z - 1 &= \ln(x)
            \\
            \exp(z - 1) &= x
        \end{align*}
        No edge cases, this is well-defined for $x > 0$. 
        Substituting it back to the expression inside of sup, we have: 
        $$
        f^*(z) = z\exp(z - 1) - (z - 1)\exp(z - 1)
        $$
\section*{Problem (2)}
    \subsection*{(a)}
        \textbf{Objective: }Express the conjugate of the function $f(x):= \lambda g(x)$ in terms of the conjugate of $g(x)$. 
        \begin{align*}\tag{2a1}\label{eqn:2a1}
        f^*(z) &= \sup_x \left\lbrace
            z^Tx - \lambda g(x)
        \right\rbrace
        \\
        &= \lambda \sup_x \left\lbrace
            \frac{z^T}{\lambda}x - g(x)
        \right\rbrace
        \\ 
        &= \lambda g \left(
            \frac{z}{\lambda}
        \right)
        \end{align*}
    \subsection*{(b)}
        \textbf{Objective:} $f(x):= g(x - a) + b^Tx$ find conjugate of $f(x)$ in term of conjugate of $g$. 
        \begin{align*}\tag{2b1}\label{eqn:2b1}
            f^*(z) &= \sup_x \left\lbrace
                z^Tx - g(x - a) - b^Tx
            \right\rbrace 
            \\
            &= \sup_x \left\lbrace
                (z^T - b^T)x - g(x - a)
            \right\rbrace \quad \text{let: } y = x - a
            \\
            &= \sup_{y + a} \left\lbrace
                (z^T - b^T)(y + a) - g(y)
            \right\rbrace
            \\
            &= 
            \sup_{y + a} \left\lbrace
                (z^T - b^T)y + g(y)
            \right\rbrace + (z^T - b^T)a
            \\
            &= g^*(y) + (z^T - b^T)a
        \end{align*}
    \subsection*{(c)}
        \textbf{Objective:} Figure out conjugate of $f(x):= \inf_z \{g(x, z)\}$ using conjugate of $g(x, y)$. 
        \\
        The format of taking the conjugate of a bi-variable function is hinted in the lecture, and let's define it to be: 
        $$
            g^*(w, z) = \sup_{x, y} \left\lbrace
                w^Tx + z^Ty - g(x, y)
            \right\rbrace
        $$
        So then we can start with the definition of the conjugate of $f$: 
        \begin{align*}\tag{2c1}\label{eqn:2c1}
            f^*(y) &= \sup_x \left\lbrace
                y^Tx - \inf_z \left\lbrace
                    g(x, z)
                \right\rbrace
            \right\rbrace
            \\
            &= \sup_x \left\lbrace
                y^Tx + \sup_z \left\lbrace
                    -g(x, z)
                \right\rbrace
            \right\rbrace
            \\
            &= \sup_{x, z} \left\lbrace
                y^Tx + w^T\mathbf{0} - g(x, z)
            \right\rbrace
            \\
            &= g^*(y, \mathbf{0})
        \end{align*}
    \subsection*{(d)}
        \textbf{Objective: }Find the conjugate of $f(x):= \inf_z \{
            \frac{1}{2}\Vert x - z\Vert^2 + g(z)
        \}$
        \\
        Notice that, the expression fits the pattern described in (c), and therefore, we are interested in looking for the conjugate of the function inside the inf: 
        $$
        h(x, y) := \frac{1}{2} \Vert x - y\Vert^2 + g(y) \quad f(x) = \inf_z \left\lbrace
            h(x, z)
        \right\rbrace
        $$
        Let's work it out: 
        \begin{align*}\tag{2d1}\label{eqn:2d1}
            h^*(v, w) &= \sup_{x, y} \left\lbrace
                \underbrace{v^Tx + w^Ty - \frac{1}{2}\Vert x - y\Vert^2 - g(y)}_\text{smooth wrt x}
            \right\rbrace
            \\
            \nabla_x \left[ 
                v^Tx + w^Ty - \frac{1}{2}\Vert x - y\Vert^2 - g(y)
            \right] &= \mathbf{0}
            \\
            v - (x - y) &= \mathbf{0}
            \\
            x &= v + y
            \\
            h^*(v, w) &= \sup_y \left\lbrace
                v^T(v + y) + w^t y - \frac{1}{2} \Vert v\Vert^2 - g(y)
            \right\rbrace
            \\
            &=
            \sup_y \left\lbrace
                (v^T + w^T)y - g(y)
            \right\rbrace + \frac{1}{2}\Vert v \Vert^2
            \\
            &= g^*(v + w) + \frac{1}{2} \Vert v\Vert^2
        \end{align*}
        Using \hyperref[eqn:2c1]{2c1}, we have: 
        \begin{align*}\tag{2d2}\label{eqn:2d2}
            f^*(y) &= h^*(y, \mathbf{0}) 
            \\
            f^*(y) &= g^*(y) + \frac{1}{2}\Vert y\Vert^2
        \end{align*}
\section*{Problem (3)}
    \subsection*{(a)}
        \textbf{Objective:} Derive the equality: $$
        \underset{f}{\text{prox}} \left(
            z
        \right) + 
        \underset{f^*}{\text{prox}}(z) = z
        $$
        With the assumption the $f$ is closed and convex. 
        \\
        \textbf{Claim 3a1\label{3a1}}
        $$
            u = \underset{f}{\text{prox}}(z) \iff z - u\in \partial f(u)
        $$
        \textbf{Claim 3a2 \label{3a2}}
        $$
        z \in \partial f(x) \iff x\in \partial f^*(z)
        $$
        \textbf{Proof:}
        \begin{align*}\tag{3a3}\label{eqn:3a3}
            z-u &\in \partial (u) \\
            u &\in \partial f^*(z - u) \quad \text{by \hyperref[3a2]{3a2}}\\
            z - (z - u) &\in \partial f^*(z - u) \\
            z - u &= \underset{f^*}{\text{prox}}(z) \quad \text{by \hyperref[3a1]{3a1}}
        \end{align*}
        Therefore: 
        $$
        \underset{f}{\text{prox}} \left(
            z
        \right) + 
        \underset{f^*}{\text{prox}}(z) = z - u + u = z
        $$

    \subsection*{(b)}
        \textbf{Objective}: Verify part (a) using $\text{prox}_{\Vert \cdot\Vert_1},\text{prox}_{\Vert \cdot\Vert_2}$
        From part (1)(a), we conclude that: 
        $$
        \delta_{\mathbb{B}_\infty}^*(x) = \delta_{\mathbb{B}_1}(x)
        $$
        From previous HW2 we gathered that: 
        $$
        \left(
            \underset{1, \Vert \cdot\Vert_1}{\text{prox}}(y)
        \right)_i
        =
        \begin{cases}
            0 & y \in [-1, 1]
            \\
            y_i + 1 & y_i < -1
            \\
            y_i - 1 & y_i > 1
        \end{cases}
        $$
        $$
        \left(
            \underset{1, \Vert \cdot\Vert_\infty}{\text{prox}}(y)
        \right)_i
        = 
        \begin{cases}
            y_i & y_i \in [-1, 1] \\
            -1 & y_i < -1 \\
            1 & y_i > 1
        \end{cases}
        $$
        If I add them together, element-wise, each case matches each other, cancelling out, and then we just have the vector $y$. 
        \\
        For the second part, we need the prox with the $\Vert \cdot\Vert_2$, and the conjugate of $\Vert \cdot\Vert_2$ to verify. 
        \\
        From \hyperref[eqn:1b1]{1b1} we have: 
        $$
            (\delta_{\mathbb{B}_2}(x))^* = \Vert x\Vert_2
        $$
        From Previous HW we know that: 
        $$
        \underset{\Vert \cdot\Vert}{\text{prox}}(z) 
        =
        \begin{cases}
            (\Vert z\Vert_2 - t)\widehat{z} & \Vert z\Vert_2 > t 
            \\
            \mathbf{0} & \Vert z\Vert_2 \le t
        \end{cases}
        $$
        And the prox for the $\delta_{\mathbb{B}_2}$ is like: 
        $$
        \underset{\delta_{\mathbb{B}_2}}{\text{prox}}(z) = \underset{\mathbb{B}_2}{\text{proj}} (z)
        = \begin{cases}
            \widehat{z} & \Vert z\Vert_2 \ge 1
            \\
            z & \text{else}
        \end{cases}
        $$
        Adding then case by case we have: 
        $$
        (\Vert z\Vert_2 - t)\widehat{z} + \widehat{z} = \Vert z\Vert_2 \widehat{z} = z \quad \mathbf{0} + z = z
        $$
        Under both cases, the identity is verified. 


\section*{Problem (4)}
    \subsection*{(a)}
    

    
    
    
\end{document} 